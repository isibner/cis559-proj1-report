%%%%%%%%%%%%%%%%%%%%%%%%%%%%%%%%%%%%%%%%%
% Arsclassica Article
% LaTeX Template
% Version 1.1 (10/6/14)
%
% This template has been downloaded from:
% http://www.LaTeXTemplates.com
%
% Original author:
% Lorenzo Pantieri (http://www.lorenzopantieri.net) with extensive modifications by:
% Vel (vel@latextemplates.com)
%
% License:
% CC BY-NC-SA 3.0 (http://creativecommons.org/licenses/by-nc-sa/3.0/)
%
%%%%%%%%%%%%%%%%%%%%%%%%%%%%%%%%%%%%%%%%%

%----------------------------------------------------------------------------------------
%	PACKAGES AND OTHER DOCUMENT CONFIGURATIONS
%----------------------------------------------------------------------------------------

\documentclass[
10pt, % Main document font size
letterpaper, % Paper type, use 'letterpaper' for US Letter paper
oneside, % One page layout (no page indentation)
%twoside, % Two page layout (page indentation for binding and different headers)
headinclude,footinclude, % Extra spacing for the header and footer
english
]{article}

\input{structure.tex} % Include the structure.tex file which specified the document structure and layout
\usepackage[letterpaper]{geometry}
\geometry{verbose,tmargin=1in,bmargin=1.5in,lmargin=1.5in,rmargin=1.5in}

\hyphenation{Fortran hy-phen-ation} % Specify custom hyphenation points in words with dashes where you would like hyphenation to occur, or alternatively, don't put any dashes in a word to stop hyphenation altogether

%----------------------------------------------------------------------------------------
%	TITLE AND AUTHOR(S)
%----------------------------------------------------------------------------------------

\title{\normalfont\spacedallcaps{CIS 559 Project 1:\break Parallel Football}} % The article title

\author{\spacedlowsmallcaps{Ian Sibner, Derick Olson, Spriha Baruah, Anthony Hsieh}} % The article author(s) - author affiliations need to be specified in the AUTHOR AFFILIATIONS block

\date{19 September 2015} % An optional date to appear under the author(s)

%----------------------------------------------------------------------------------------

\begin{document}

%----------------------------------------------------------------------------------------
%	HEADERS
%----------------------------------------------------------------------------------------

\renewcommand{\sectionmark}[1]{\markright{\spacedlowsmallcaps{#1}}} % The header for all pages (oneside) or for even pages (twoside)
%\renewcommand{\subsectionmark}[1]{\markright{\thesubsection~#1}} % Uncomment when using the twoside option - this modifies the header on odd pages
\lehead{\mbox{\llap{\small\thepage\kern1em\color{halfgray} \vline}\color{halfgray}\hspace{0.5em}\rightmark\hfil}} % The header style

\pagestyle{scrheadings} % Enable the headers specified in this block

%----------------------------------------------------------------------------------------
%	TABLE OF CONTENTS & LISTS OF FIGURES AND TABLES
%----------------------------------------------------------------------------------------

\maketitle % Print the title/author/date block

\setcounter{tocdepth}{2} % Set the depth of the table of contents to show sections and subsections only

\tableofcontents % Print the table of contents

\listoffigures % Print the list of figures

% \listoftables % Print the list of tables

%----------------------------------------------------------------------------------------
%	ABSTRACT
%----------------------------------------------------------------------------------------

\section{Introduction} % This section will not appear in the table of contents due to the star (\section*)

\lipsum[1] % Dummy text

\section{Initial Insights and Observations}

\lipsum[3]

\section{Strategies and Concepts}
\subsection {Global scope}

Given the loose efficiency constraints given the problem, we decided that a globally optimal benefit function would be both feasible and desirable. The primary gain from this decision was a team optimized for late gameplay, where our players were able to seek out and steal balls toward the end of the game. The reason was that once we and other teams cleared the balls near our respective goals, balls became scarce. If any team had accumulated balls, even if they were far from our players, they would seek them out and attempt to score with them.

On the other hand, as teams improved, the game time grew shorter and quicker. In such a game, it is possible that a globally optimal zone will change into a depleted zone before a far-away team can reach it. It is likely that our model did not fully capture the constantly changing state of the game board. We addressed this by heavily weighting our own goal in benefit calculations.

\subsection{Benefit Function}

\subsection{Cell clustering}
On the second iteration of our benefit function, we accounted for ball and opponent clustering by keeping track of the number of balls and opponents surrounding a particular cell. This allowed our players to seek out areas of high benefit, rather than single cells. 

After initial trials with a large radius, we settled on a radius of 3. We observed that larger radii caused the players to move back and forth somewhat inefficiently, and hypothesize that these larger zones changed too quickly for players to capitalize on them.

\subsection{Kicking}
Initially, our players would kick the maximal distance towards the goal. Although this will guarantee that the ball will be as close to the goal as possible, it does not necessarily mean that the number of steps to kick that ball in the goal is minimized when there are teamates around. With this insight, we improved the kicking algorithm with a simple calculation. A player will look at all possible cells it can kick to and minimize the number of steps to kick that ball into the goal for anyone on its team. What this translates to in practice is that a player will now pass the ball as close to a teammate as possible if the teammate is closer to the goal than itself. We observed that after implementing this kicking algorithm, the players naturally form a bucket brigade to minimize the number of steps a player needs to go to for the ball to score.

\section{Implementation}

\subsection{Locking Mechanism}
As we found through our trials, using a benefit function to determine where a player should go runs into problems when players want to go to the same cell. The players will eventually converge and go to the same cells the rest of the game. To prevent this major inefficency, we implemented a locking mechanism where only one player can go after a single cell at a time. However, we found that although this works great through most of the game, when the number of balls is very small, it becomes better for multiple players to go after the same ball to have a higher likelyhood of kicking the ball to the goal. As such, when the number of balls reaches the number of players on the field, we turn off the locking mechanism.

After only locking the cell that the player is going towards, we also lock a radius around that cell. This allows us to create zones such that only one player clearing a zone.Through experimentation, we found that choosing a radius of max(0,3-P/10) works well. With higher number of players, the smaller the zones should become because there is not enough space to create so many zones. Addditionally, it is more likely that opponents will steal balls from a particular zone before a single player can clear that zone out.

\subsection {Auxiliary Board}

Reference to Figure~\vref{fig:gallery}. % The \vref command specifies the location of the reference

\begin{figure}[tb]
\centering 
\includegraphics[width=0.5\columnwidth]{GalleriaStampe} 
\caption[An example of a floating figure]{An example of a floating figure (a reproduction from the \emph{Gallery of prints}, M.~Escher,\index{Escher, M.~C.} from \url{http://www.mcescher.com/}).} % The text in the square bracket is the caption for the list of figures while the text in the curly brackets is the figure caption
\label{fig:gallery} 
\end{figure}

\section{Results}

\lipsum[3]

\section{Contributions}

\lipsum[2]

\section{Future Directions and Limitations}

\subsection{Generalized Step Distance}
This idea stems from our attempts to take advantage of global-scope “hotspots,” or high-density ball clusters that are not near the goal. The existing implementation weights balls close to the home goal particularly high, which is good in the early game, but may not be ideal overall. 

As we can see in the figure above, certain strategies tend to cluster balls near an opponent goal, where they stay relatively untouched for several cycles. A smarter global-cluster detection would drop the current strategy 

The biggest disadvantage to this approach is that the time it takes to travel to such a cluster may be greater than the the time the cluster exists. Even if the seeking team is able to reach the cluster in time, the balls lost in the transit time may not be worth it.

It would be possible to determine, by factoring in the number of balls left $B_{total}$, the number of balls in close range $B_{nearby}$, and the number of balls expected to remain in the cluster $B_{cluster}$. Such a strategy would decide to go for the cluster if and only if: 
$$B_{total} - B_{nearby} < B_{cluster}$$

One way of implementing this strategy would be to generalize the  \texttt{numberOfStepsToGoal()} benefit function to be \texttt{numberOfStepsToPosition(p)}, where the latter takes in any valid position $p$ as it’s argument, causing the cells around it to get a score boost. Then, upon deciding to pursue a cluster, this position would be updated to the cluster center, until a new position was found.

\subsection{Pathing}
It becomes clear through watching our players that a pathing algorithm could improve our players drastically. We found that our players will often take the same path to their desired locations even though their destinations are distinct. This creates inefficiencies as if there are balls on that path, only one player is needed to clear those balls. In the future, we would look into minimizing the number of cells such that two players would through to reach their destination. 

Another problem with our players with pathing is that they arbitrarily choose a minimum path to its destination. So, if two paths are equally minimal, the player will not choose the path with the most balls, thus missing opportunties to score more points. In the same vein, if two paths are slightly different in the number of steps but the longer path has way more balls, the player will choose the shorter path, causing inefficiencies. To fix this problem, The path the player should take should additionally take into consideration how many balls the player can kick on that path.

\section{Acknowledgments}

Our progress was largely a result of the class discussions and adopted strategies. Through the course of the project, we adopted starting strategies from Groups 1 (Grid Player), as well as experiments in stealing opponent balls with the opening placements. We would like to give special thanks to Groups 2 and 6 for the important heuristics we adopted from them and eventually used in our final implementation. 

Group 2 (The Swarm) was the first group to successfully implement a kind of benefit function with the emergent behavior of a bucket brigade. This function was based on the number of steps from the ball to the goal, and we adopted it as a core feature of our ensemble benefit function.

Group 6 (Bucket Brigade) contributed a smarter kicking function that favored passing to players over kicking as far to the goal as possible. We adopted this idea as a core part of our improved kicking strategy.

\section{Conclusion}

\lipsum[1]

\subsection{Figure Composed of Subfigures}

Reference the figure composed of multiple subfigures as Figure~\vref{fig:esempio}. Reference one of the subfigures as Figure~\vref{fig:ipsum}. % The \vref command specifies the location of the reference

\begin{figure}[tb]
\centering
\subfloat[A city market.]{\includegraphics[width=.45\columnwidth]{Lorem}} \quad
\subfloat[Forest landscape.]{\includegraphics[width=.45\columnwidth]{Ipsum}\label{fig:ipsum}} \\
\subfloat[Mountain landscape.]{\includegraphics[width=.45\columnwidth]{Dolor}} \quad
\subfloat[A tile decoration.]{\includegraphics[width=.45\columnwidth]{Sit}}
\caption[A number of pictures.]{A number of pictures with no common theme.} % The text in the square bracket is the caption for the list of figures while the text in the curly brackets is the figure caption
\label{fig:esempio}
\end{figure}

%----------------------------------------------------------------------------------------

\end{document}
